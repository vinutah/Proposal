\documentclass[10pt, oneside]{article}
\usepackage[letterpaper, margin=1in]{geometry}
%\usepackage[parfill]{parskip}    		% Activate to begin paragraphs with an empty line rather than an indent
\usepackage{graphicx}
\usepackage{amssymb}
\usepackage[style=numeric,firstinits=true,backend=bibtex]{biblatex}
\addbibresource{main}     %Adds the bib file (bib-database)
\usepackage{wrapfig}
\usepackage{algorithm}
\usepackage{algpseudocode}      % for writing pseudocode
\usepackage{amsmath}
\usepackage{amsfonts}
\usepackage{url}
% \usepackage{tikz}              % for drawing superb diagrams, use omni-graffle or yEd
% \usetikzlibrary{calc,shapes.multipart,chains,arrows}
\usepackage{graphicx}
\usepackage{nameref}            % Cross-referencing to include the name of the section, 
                                % rather than just the number or page

% \usepackage{adjustbox}          %Adjustments like in "includegraphics" options
\usepackage{framed}             % Create framed, shaded, or differently highlighted 
                                % regions that can break across pages
\usepackage[all]{xy}
\usepackage{txfonts,pxfonts}    %additional fonts: for more symbols
\usepackage{bm}
\usepackage{enumerate}          %gives the enumerate environment an optional argument
                                % which determines the style in which the counter is 
                                % printed
\usepackage{subfigure}          %for sub figures
\usepackage[linktoc=all]{hyperref}
\usepackage{enumitem}
\usepackage{multirow}
\setlength{\parskip}{.25ex plus1mm minus1mm}
% ----------------------------------------
%  Nice code using inconsolata font
% ----------------------------------------

%\usepackage{listings}
\usepackage{xcolor}

% Code listing stuff
\usepackage{listings}
\usepackage[T1]{fontenc}
\usepackage[utf8]{inputenc}
\usepackage[scaled=0.8]{beramono}
\usepackage{amsmath}
\usepackage{amsfonts}
\usepackage{amssymb}
\definecolor{light-gray}{gray}{0.45}
\definecolor{dark-green}{rgb}{0.0,0.34,0.25}

\lstset { language=Python }

\lstset { 
	% Basic formatting
	basicstyle=\ttfamily\footnotesize,
	keywordstyle=\color{blue}\ttfamily\bfseries,
	commentstyle=\color{light-gray}\ttfamily\itshape,
	breaklines=true,
	escapechar=\%,
	showspaces=false,
	captionpos=b,
	%abovecaptionskip=\medskipamount,
	%belowskip=-2em,
	% Line numbering
	numbers=left,
	xleftmargin=3.6em,
	framexleftmargin=1.5em,
	numberstyle=\color{light-gray},
	firstnumber=1,
	numberfirstline=true
}


%\usepackage{fontspec}
%\setmonofont{Consolas}

%\lstdefinelanguage{ds2}{
%  morekeywords={
%    world, machine, agents, compute, %
%    agents-bandwidth, agents-reactions , %
%    level, parent, sync, async, send, broadcast,%
%    reactions, sender,
%    schedule, function,%
%    on-join, on-rejoin, on-leave, on-start, on-demise, on-link-failed,%
%    abstract,case,catch,class,def,%
%    do,else,extends,false,final,finally,%
%    for,if,implicit,import,match,mixin,%
%    new,null,object,override,package,%
%    private,protected,requires,return,sealed,%
%    super,this,throw,trait,true,try,%
%    type,val,var,while,with,yield},
%  otherkeywords={!,:,=>,<-,<\%,<:,>:,\#,@},
%  sensitive=true,
%  morecomment=[l]{//},
%  morecomment=[n]{/*}{*/},
%  morestring=[b]",
%  morestring=[b]',
%  morestring=[b]"""
%}
%
%\lstset{ %
%  postbreak=\raisebox{0ex}[0ex][0ex]{\ensuremath{\color{red}\hookrightarrow\space}}, % shows hook arrow and indents wrapped line 
%  backgroundcolor=\color{white},   % choose the background color; you must add \usepackage{color} or \usepackage{xcolor}
%  basicstyle=\tiny,        % the size of the fonts that are used for the code
%  breakatwhitespace=false,         % sets if automatic breaks should only happen at whitespace
%  breaklines=true,                 % sets automatic line breaking
%  captionpos=b,                    % sets the caption-position to bottom
%  commentstyle=\color{olive},    % comment style
%  deletekeywords={...},            % if you want to delete keywords from the given language
%  escapeinside={\%*}{*)},          % if you want to add LaTeX within your code
%  extendedchars=true,              % lets you use non-ASCII characters; for 8-bits encodings only, does not work with UTF-8
%  frame=single,	                   % adds a frame around the code
%  keepspaces=true,                 % keeps spaces in text, useful for keeping indentation of code (possibly needs columns=flexible)
%  keywordstyle=\color{blue},       % keyword style
%  language=ds2,                 % the language of the code
%  otherkeywords={*,...},           % if you want to add more keywords to the set
%  numbers=left,                    % where to put the line-numbers; possible values are (none, left, right)
%  numbersep=5pt,                   % how far the line-numbers are from the code
%  numberstyle=\tiny\color{gray}, % the style that is used for the line-numbers
%  rulecolor=\color{gray},         % if not set, the frame-color may be changed on line-breaks within not-black text (e.g. comments (green here))
%  showspaces=false,                % show spaces everywhere adding particular underscores; it overrides 'showstringspaces'
%  showstringspaces=false,          % underline spaces within strings only
%  showtabs=false,                  % show tabs within strings adding particular underscores
%  stepnumber=2,                    % the step between two line-numbers. If it's 1, each line will be numbered
%  stringstyle=\color{red},     % string literal style
%  tabsize=2,	                   % sets default tabsize to 2 spaces
%  title=\lstname,                   % show the filename of files included with \lstinputlisting; also try caption instead of title
%}

% ----------------------------------------

% \raggedbottom % less spacing between items 

% Recommended, but optional, packages for figures and better typesetting:
\usepackage{microtype}
\usepackage{graphicx}
\usepackage{subfigure}
\usepackage{booktabs} % for professional tables
\usepackage{tabularx}
\usepackage{upgreek}
\usepackage{comment}
\usepackage{amsmath,array,graphicx}

\usepackage{tabularx}
\usepackage{multirow}
\usepackage{graphicx}
\usepackage{algorithm}
%\usepackage[noend]{algpseudocode}
\usepackage[colorinlistoftodos]{todonotes}
%\usepackage{algorithm,algpseudocode}
\usepackage{upgreek}

% hyperref makes hyperlinks in the resulting PDF.
% If your build breaks (sometimes temporarily if a hyperlink spans a page)
% please comment out the following usepackhttps://www.overleaf.com/project/5cedce0cdfbeaa5a7375b030age line and replace
% \usepackage{sysml2019} with \usepackage[nohyperref]{sysml2019} above.
%\usepackage{hyperref}

% Attempt to make hyperref and algorithmic work together better:
\newcommand{\theHalgorithm}{\arabic{algorithm}}

% Code listing stuff
\usepackage{listings}
\usepackage[T1]{fontenc}
\usepackage[utf8]{inputenc}
\usepackage[scaled=0.8]{beramono}
\usepackage{amsmath}
\usepackage{amsfonts}
\usepackage{amssymb}
\usepackage{xcolor}
\usepackage{xspace}
\definecolor{light-gray}{gray}{0.45}
\definecolor{dark-green}{rgb}{0.0,0.34,0.25}

\title{\huge \textbf{Systems and Methods for Resource Efficient \\ and Reliable Neural Network Compression}}
%
\author{\\
        \\\huge \textbf{Dissertation Proposal} \\
        \\
        \\
        \\\huge \textbf{Candidate} \\
        \\
        \\\huge Vinu Joseph \\
        \\ \LARGE \texttt{vinu@cs.utah.edu}
        \\
        \\
        \\
        \\
        \\\huge \textbf{Dissertation Committee}\\
        \\
        \\\huge Ganesh Gopalakrishnan \\
        \\\huge Aditya Bhaskara \\
        \\\huge Vivek Srikumar \\
        \\\huge Mary Hall \\
        \\\huge Michael Garland \\
        
        }
\date{}
%
%\date{\LARGE October 09, 2018 -- October 16, 2018}
%        
%
%\title{\textbf{PhD Written Qualifier Exam}
%\vspace{5ex}
%\\
%\underline{Examiners}
%\\ 
%Ganesh Gopalakrishnan
%\\
%Mary Hall
%\\
%Aditya Bhaskara
%\\
%Vivek Srikumar}
%%
%\author{\underline{Examinee}\\ Vinu Joseph\\{\texttt{vinu@cs.utah.edu}}}
%
%
\renewcommand*{\bibfont}{\footnotesize}
\lstset{basicstyle=\ttfamily,
escapeinside={||},
mathescape=true}
%
\newenvironment{blockquote}{\par\medskip\leftskip=4em\rightskip=2em\noindent\ignorespaces}{\par\medskip}
%
%\DeclareMathOperator*{\argmax}{arg\,max}
%\DeclareMathOperator*{\argmin}{arg\,min}
\newcommand{\norm}[1]{\left\lVert#1\right\rVert}
\newcommand{\algoName}[0]{\textsc{Condensa}\xspace}
%
\begin{document}
\maketitle
\newpage
\setcounter{tocdepth}{3}
\tableofcontents
\newpage

%\begin{refsection}
\section{Introduction}


Modern deep neural networks (DNNs) are complex,
and often contain millions of parameters spanning
dozens or even hundreds of layers~\cite{he2016deep,huang:2017}.
%
This complexity translates into substantial memory and runtime costs
on hardware platforms at all scales.
%
Recent work has demonstrated that DNNs are often over-provisioned and can be compressed without appreciable loss of accuracy.
Model compression can be used to
reduce both model memory footprint and inference latency using techniques such as
weight pruning~\cite{han2015learning,luo2017thinet},
quantization~\cite{gupta2015deep}, and low-rank
factorization~\cite{jaderberg2014speeding,denton2014exploiting}.
%
Unfortunately, the requirements of
different {\em compression contexts}---DNN structure,
target hardware platform, and the user's optimization objective---are often in conflict.
%
The recommended compression strategy for reducing inference latency
may be different from that required to reduce total memory footprint.
%
For example, in a Convolutional Neural Network (CNN),
reducing inference latency may require pruning filters to realize speedups on real hardware~\cite{li2016pruning}, while reducing memory footprint may be accomplished by zeroing out individual weights.
%
Similarly, even for the {\em same optimization objective},
say reducing inference latency, one may employ filter pruning for a CNN,
while pruning 2-D blocks of non-zero weights~\cite{gray:2017} for a
language modeling network such as Transformer~\cite{vaswani:2017},
since the latter has no convolutional layers.
%
Thus, it is crucial to enable convenient expression of 
alternative compression schemes, yet
none of today's model compression approaches help the designer
tailor compression schemes to their needs.


\begin{figure}[!h]
\centering
%\fbox{\rule{0pt}{2in} \rule{0.9\linewidth}{0pt}}
\includegraphics[width=0.5\linewidth]{img/vgg19_bn_filter_intro_v2.pdf}
%\vspace{-10pt}
\caption{Top-1 test accuracy (green) and throughput (red) vs.\ sparsity for VGG-19 on CIFAR-10.
\algoName is designed to solve constrained optimization problems of the form ``maximize throughput, with a lower bound on accuracy". In this case, \algoName automatically discovers a sparsity (vertical dashed line) and compresses the model to this sparsity,
improving throughput by $2.59\times$ and accuracy by $0.36\%$.
}
\vspace{-10pt}
\label{fig:vgg-intro}
\end{figure}

%Current approaches to model compression
%also require manual specification of compression hyperparameters, such
%as the target sparsity ratio, which is the proportion of zero-valued parameters in %the
%compressed model vs.\ the original.
%
Current approaches to model compression
also require manual specification of compression hyperparameters, such
as {\bf target sparsity}---{\em the proportion of zero-valued parameters in the
compressed model vs.\ the original.}
%
However, with current approaches, finding the best sparsity 
often becomes a trial-and-error search, with
each such trial having a huge cost (often multiple days for large models such as BERT) and involving training the compressed model to convergence,
only to find (in most cases) that the compression objectives are not met.
%
The main difficulty faced by such unguided approaches is
that sparsities 
vary unpredictably with changes in the compression context,
making it very difficult to provide users with any guidelines, whatsoever.
%
Therefore, automatic and {\em sample-efficient} approaches that minimize the number of trials are crucial
to support the needs of designers who must adapt
a variety of neural networks to a broad spectrum of platforms targeting a wide
range of tasks.

To address the above-mentioned problems of flexible expression of compression strategies, automated compression hyperparameter inference, and sample efficiency, we introduce \algoName, a new framework for programmable model compression. As an illustration of the level of automation provided by \algoName,
consider the problem of improving the
inference throughput of VGG-19~\cite{simonyan2014very} on the CIFAR-10 image
classification task~\cite{krizhevsky2014cifar}.
%
Since VGG-19 is a convolutional neural network, one way to improve its
inference performance on modern hardware such as GPUs is by pruning
away individual convolutional filters~\cite{he2018progressive}.
%
Figure~\ref{fig:vgg-intro} shows the accuracy and throughput obtained
by \algoName on this task.
%
Here, we plot the compressed model's top-1 test accuracy and throughput as a function of the sparsity (green and red lines,
respectively).\footnote{Note that these curves are not known a priori and
are often extremely expensive to sample;
they are only plotted here to better place the obtained solution in context.}
%
\algoName's solution corresponds to a sparsity of $0.79$
and is depicted as the vertical dashed line.
%
This result is significant for two reasons: (1) using the \algoName library,
the filter pruning strategy employed for this experiment was expressed in
less than 10 lines of Python code, and (2) the optimal sparsity of
$0.79$ that
achieves throughput and top-1 accuracy improvements of $2.59\times$ and $0.36\%$, respectively,
was obtained automatically by \algoName using a sample-efficient constrained
Bayesian optimization algorithm.
%
Here, the user didn't have to specify any
sparsities manually, and instead only had to define a domain-specific
objective function to maximize (inference throughput, in this case).

\begin{figure*}[!t]
\centering
  % \includegraphics[width=0.8\textwidth,height=10cm]{CVPR2021/img/FinalIntroFigure.pdf}
  \includegraphics[width=\textwidth]{img/FinalIntroFigure.pdf}
  \caption{Overview of our CIE reduction framework using label preservation-aware loss functions.}
  \label{fig:overview}
\end{figure*}

%\end{refsection}
%\begin{refsection}

\section{Thesis Statement}
\label{sec:condensa}

%\end{refsection}
\begin{refsection}
\section{Research Work}
\label{sec:work}

\input{inputs/condensa}
\subsection{Sample Efficient Methods for Compression Hyperparameter Search}
\subsubsection{Background, Related Work}
\subsubsection{Contribution}
\input{inputs/reliability}
\subsection{Proposed Work}

\subsubsection{Class-wise CIEs: Mitigating The Bias problem for Fairness in Edge Inference}

\subsubsection{Attribution Matching for Sensitive Edge Inference Tasks}

Mitigating the impact of compression is particularly
urgent given the widespread use of compressed deep 
neural networks in resource constrained but sensitive 
domains such as 
%
%
%
health care diagnostics 
\cite{xie2019automated, %(Xie et al., 2019; 
gruetzemacher20183d,    %Gruetzemacher et al., 2018; 
badgeley2019deep,       %$Badgeley et al., 2019;
oakden2020hidden}       %Oakden-Rayner et al., 2019),
%
%
,self-driving cars 
\cite{teslacrash17} %(NHTSA, 2017)
%
facial recognition software
\cite{
buolamwini2018gender% Buolamwini
 % Gebru, 2018b). 
} and
%
hiring
\cite{amazon18, 
yourface19}.% Harwell2019
%
%
For these tasks, the trade-offs incurred by 
compression will be intolerable given the huge impact 
on human welfare.
%
\cmt{Due to the success of Deep learning, there is an 
emergent trend to utilize deep neural networks (DNNs) 
even for safety-critical applications such as 
self-driving cars and health-care applications 
\cite{estava2017dermatologist,
samala2018evolutionary, lane2018deep}}.
%
Due to the inherent nature of such devices, 
it is of paramount importance that the utilized 
DNNs be reliable and trustworthy to human users.
%

For a system to be reliable, perpetual service 
must be rendered and the integrity of the system 
must hold even under unexpected circumstances.
%
%
For most commercially deployed DNNs, this condition 
is hardly met as they are often operated in the 
cloud due to their heavy computational requirements.
%
%
However, this dependence on clouds acts as a 
critical weakness in safety-sensitive settings 
as intermittent communication failuers to the 
cloud may cause difficulties in reacting to 
situations immediately, or even-worse, 
the device's connection to the cloud may be 
severed indefinitely.
%
%
Thus, to guarantee reliable service, 
the DNNs must be embedded on the edge device.
%
%
%
To this end, network compression techniques such 
as pruning \cite{han2015deep,li2016pruning} and distillation \cite{hinton2015distilling,zagoruyko2016paying} 
are commonly employed - as a compressed network 
would require less computational time and memory 
but maintain its prediction performance to a 
certain acceptable margin, effectively substituting 
the original network for edge computation.

\vcmt{This text is for attribution-perservation, we need to rewrite for label-preservation}
\cmt{At the same time, for a system to be trustworthy, 
the system must be transparent enough for humans 
to understand its workings and the reasons for its outputs.
%
%
An example would be when a health monitor 
predicts an onset of disease \cite{xu2019current}
- then the clinician would require an acceptable
explanation to the device output.
%
%
However, the black-box nature of deep neural 
networks complicates this goal - impeding its 
advance in safety-critical areas.
%
%
For DNNs to gain trustworthiness, the ability 
to explain why the network makes such decisions 
is essential. Such field of interest - 
eXplainable AI (XAI) - has emerged as one of the 
importance frontiers in the field of deep learning.
%
%
Amoung numerous XAI methods, the most commonly 
used methods are attribution methods \cite{selvaraju2017grad}, 
which weigh the parts of the input data according to 
how much they 'contributed' to produce the output prediction.
%
%
Such attribution methods are beginning to be applied
in safety-critical fields \cite{liang2020prediction}.}

To ensure the safety of the system, 
the two aforementioned conditions should be simultaneously 
satisfied - the embedded DNNs must be equipped with 
both compression and attribution.
%
%
However we show for the first time that these seemingly
unrelated techniques conflict with each other:
\vcmt{compressing a network causes deformations in 
the produced attributions, even if the predictions 
of the network stays the same before and after compression.}
%
%
\vcmt{This is a potentially severe crack in the 
integrity of the compressed network, 
as the premise in which a compressed network 
is acceptable in safety-critical fields is that 
the compressed network in as reliable as its former self}.
%
%
\textit{This implies that the compressed nework must behave 
almost identically to the pre-compression network while being
smaller in size.}
%
%
Moreover, the classifications between the network are
not only different from their past counterparts 
but also broken down compared to their respective ground truths
%
%
\begin{itemize}
    \item Examples where the compressed model gets the example 
          right but the uncompressed model gets it wrong.
    \item Examples where the uncompressed model gets it right 
          but the compressed model gets it wrong.
\end{itemize}
%
%
These label distortions directly
cause incorrect interpretations, which could lead to dire 
consequences for safety-critical systems.
%
%
Such a problem arises from the pitfall of existing 
network compression approaches: they only
aim to maintain the prediction quality of the network while 
reducing the size of the network.
%
%
Compressing a network forces the network to cram its 
necessary decision procedures and information inside a
smaller space.
%
This space restriction forces the network to abandon its 
standard decision procedures and resort to using shortcuts 
and hints that are seemingly indecipherable to humans.
%
Thus, its decision procedures would become harder to 
interpret, which is reflected in its production of deformed
attribution maps
%
%
To resolve this newfound unintended
issue, we propose a novel label-preservation aware 
compression framework  to ensure both the reliability and 
trustworthiness of the compressed model.
%
...we concentrate on the observation that the labels 
of the pre-network (teacher) are closer to the ground truth signal 
compared to the post-network (student).
%
Thus, in the absence of ground truth signals, the labels
of the teacher can serve as a proxy. 
%
In this sense, we propose a automatically parameter 
tuned regularizer learning framework that matches the 
of the now-compressing network to its attribution maps 
before compression, transferring the attributional power 
of the pre-network to the post-network. Our work sheds 
new light on transfer learning techniques from the perspective of XAI,
as they can be re-interpreted and subsumed under our framework.



\end{refsection}
%\begin{refsection}
\section{Timeline, Remaining Work, Plans}
\label{sec:timeline}

\begin{table*}[h]
\centering
\begin{center}
%\begin{scriptsize}
\begin{sc}
\begin{tabular}{ll}
\toprule
\textbf{Papers, Patents and Presentations}                                                  & \textbf{Timeline}        \\ \midrule
Effective Paralleization of Belief Propation on the GPU                  & NVIDIA GTC 2018          \\ 
Message scheduling for performant, many-core belief propagation.         & IEEE HPEC 2019           \\ \midrule
A Programming System and Automation Libraries for Model Compression                           & Baylearn-19, NeurIPS-19, GTC-19 \\
Model Compression for Health care Devices Challenges and Opportunities    & GE HealthCare 2019       \\
A Programmable Approach for Neural Network  Compression                  & IEEE MICRO 2020          \\
Bayesian Optimization for Model Compression                              & NVIDIA 2019              \\ \midrule
Correctness-preserving Compression of Datasets and Neural Networks & Correctness SC 2020      \\
Reliable Model Compression via Label-Preservation-Aware Loss Functions   & CVPR 2020 (Under Review) \\
Mitigating the Bias \& Attribution Problems for Safety \& Fairness & KDD 2021 (Proposed Venue)                             \\ \midrule
\textbf{Dissertation Defense}                                            & \textbf{Spring 2021} \\   
\bottomrule
\end{tabular}
\label{tab:results}
  \end{sc}
%\end{scriptsize}
\end{center}
\end{table*}

%\end{refsection}
%%\begin{refsection}
\section{Broader Impact} 

Non-Vision Apps

Scene-Graph Preservation

Condensa-ZFP : Using New Loss functions

%\end{refsection}
\printbibliography

\end{document}

%%% Local Variables:
%%% mode: latex
%%% TeX-master: "root"
%%% End:
